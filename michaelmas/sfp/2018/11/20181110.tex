In general, if we consider a real Lie algebra $\fg_\RR$, such that $\dim(\fg_\RR)=D$, its Killing form $\kappa^{ab}$ will also in general be real. Treating $\kappa$ as a $D\times D$ matrix, one can then ask about the \term{signature} of the matrix, i.e. how many positive and negative eigenvalues $\kappa$ has when interpreted as a matrix, and more generally whether they are all of the same sign.
\begin{defn}
A real Lie algebra $\fg_\RR$ is of \term{compact type} if $\exists$ a basis in which
$$\kappa^{ab}=-K \delta^{ab}, K\in \RR^+.$$
\end{defn}
If we have a basis for $\fg_\RR$ with $\fg_\RR$ the real span of the generators $\set{T^a,a=1,\ldots,D}$ then we can take the \term{complexification} of $\fg_\RR$, which is defined to be the complex span of the same basis vectors,
$$\fg_\CC = \text{Span}_\CC \set{T^a,a=1,\ldots,D}.$$
However, going back to a real Lie algebra from a complex one is harder-- there may be several real Lie algebras which have the same complexification. Instead, we say that $\fg_\RR$ is a \term{real form} of $\fg_\CC$.

\begin{thm}
Every complex semi-simple Lie algebra of finite dimension has a real form of compact type.
\end{thm}
This may be helpful on the final question of Example Sheet 2.


Continuing with our discussion of root strings, we previously defined
$$S_{\alpha,\beta}=\set{\beta+\rho \alpha\ \in \Phi,\rho\in \ZZ}.$$
We then argued that for $V_{\alpha,\beta}$ the representation space of a repn $R$ of $sl(2)_\alpha$, there was a weight set
$$S_R=\set{\frac{2(\alpha,\beta)}{\alpha,\alpha}+2\rho; \beta+\rho\alpha \in \Phi}.$$

We then noted that if we conisder the irreps $R_\Lambda$ of $sl(2)_\alpha$ for some $\Lambda\in \ZZ_{\geq 0},$ we get
$$S_R=S_\Lambda=\set{-\Lambda,-\Lambda+2,\ldots,+\Lambda}.$$
The allowed values of $\rho$ are then $\rho=n\in \ZZ$ such that $n_-\leq n \leq n_+, n\pm \in \ZZ$ are some bounding values. By comparing the expression for the weight set with the minimum and maximum weights $\pm \Lambda,$ we see that
\begin{align*}
    -\Lambda&= \frac{2(\alpha,\beta)}{(\a,\a)}+2n_-\\
    +\Lambda &=\frac{2(\alpha,\beta)}{(\a,\a)}+2n_+\\
    \implies&\frac{2(\alpha,\beta)}{(\a,\a)}=-(n_+ + n_-)\in \ZZ.
\end{align*}
However, we also know that the allowed set of roots form an unbroken string,
$$S_{\a,\beta}=\set{\beta+n\alpha; n\in \ZZ, n_- \leq n \leq n_+}.$$
So this places a constraint on what the roots can be. This inner product constraint would be a lot stronger if we could guarantee the roots were real.

Let's pass for a moment to the Cartan-Weyl basis,
$$[H^i,E^\delta]=\delta^i E^\delta$$
where $i=1,\ldots, r \forall \delta \in \Phi.$ Then we write the Killing form as
$$\kappa^{ij}=\kappa(H^i,H^j)=\frac{1}{N}\Tr[\ad_{H^i}\circ \ad_{H^j}]$$
Now we remark that it would be very nice if these ad maps were mutually diagonal, since for
$$A=\begin{pmatrix}
\lambda_1^A & &\\
 & \ddots & \\
 & & \lambda_n^A
\end{pmatrix}, B=\begin{pmatrix}
\lambda_1^B & &\\
 & \ddots & \\
 & & \lambda_n^B
\end{pmatrix},
$$
the trace is given simply by 
$$\Tr[AB]=\sum_{i=1}^n \lambda_i^A \lambda_i^b.$$

%probably gotta revisit this argument
So let us rewrite the ad maps in terms of the roots (which are of course just the eigenvalues when we diagonalize both maps):
\begin{align*}
    \kappa^{ij}&=\frac{1}{N}\Tr[\ad_{H_i}\circ \ad_{H_j}]\\
    &= \frac{1}{N} \sum_{\delta\in\Phi} \delta^i \delta^j.
\end{align*}
Moreover we know that 
$$(\alpha,\beta)=\a^i \beta^j(\kappa^{-1})_{ij}=\frac{1}{N} \sum_{\delta\in \Phi} \alpha_i \delta^i \delta^j \beta_j,$$
where
$$\alpha_i \equiv (\kappa^{-1})_{ij}\alpha^j.$$

Now since $\alpha_i\delta^i=(\alpha,\delta),$ we see that
$$(\alpha,\beta)=\frac{1}{N}\sum_{\delta\in \Phi} (\a,\delta)\beta,\delta).$$
Thus the quantity
$$R_{\alpha,\beta}=\frac{2(\a,\beta)}{(\a,\a)}\in \ZZ.$$
Moreover
$$\frac{2}{(\beta,\beta)}R_{\a,\beta}=\frac{1}{N}\sum_{\delta\in \Phi}R_{\a,\delta}R_{\beta,\delta}\in \RR \implies (\beta,\beta)\in\RR \forall \beta\in \Phi.$$
We conclude that the inner product of two roots is always real,
$$(\a,\beta)\in \RR \quad \forall \a,\beta\in\Phi.$$

\subsection*{Real Geometry of Roots} Now that we know that the roots have real inner products, it makes good sense to discuss the real geometry of the dual space $\fh^*$. Let us now claim that the roots $\alpha \in \Phi$ are not only elements of $\fh^*$ but indeed span the dual space,
$$\fh^*=\text{Span}_\CC\set{\alpha\in\Phi}.$$
\begin{proof}
If the roots $\alpha$ do not span $\fh^*$, then
$\exists \lambda \in \fh^*$ with
$$(\lambda,\alpha)=(\kappa^{-1})_{ij}\lambda^i \alpha^j = \kappa^{ij} \lambda_i \alpha_j=0 \quad \forall \alpha \in \Phi,$$
i.e. another element $\lambda$ which is orthogonal to all the roots. Thus we can construct
$$H_\lambda=\lambda_i H^i \in \fh,$$
and we can compute some brackets now:
$$[H_\lambda,H]=0 \quad \forall H\in \fh$$
and
$$[H_\lambda,E^\a]=(\lambda,\a)E^\alpha =0 \quad \forall \alpha \in \Phi.$$
But this is very strange, because this means that
$[H_\lambda,X]=0\forall X \in \fg$. This means that $\fg$ has a non-trivial ideal, namely
$$\mathfrak{j}=\text{Span}_\CC\set{H_\lambda}.$$
But this would mean that $\fg$ is not simple, so we have reached a contradiction.
\end{proof}

Therefore the $r$ roots form a basis for $\fh^*$ (with complex coefficients), and we may then define the \term{real subspace},
$$\fh^*_\RR=\text{Span}_\RR\set{\alpha_{(i)};i=1,\ldots, r},$$
which is the \emph{real} span of the roots.

Since the roots span $\fh^*$, any root $\beta\in \Phi$ can be written as
$$\beta=\sum_{i=1}^r \beta^i \alpha_{(i)}$$
with $\beta\in \CC$ generically. However, if we take the inner product of $\beta$ with each of the $\alpha_{(j)}$s, we find that
$$(\beta,\alpha_{(j)})=\sum_{i=1}^r \beta^i(\alpha_{(i)},\alpha_{(j)}.$$
 However, we know that the inner products of the roots $\alpha$ are real, so $(a_{(i)},\a_{(j)})$ considered as an $r\times r$ matrix is real, and $(\beta,\alpha_{(j)}$ considered as a vector of length $r$ is also real. Therefore all the coefficients $\beta^i$ must also be real, which means that all the roots live in the real subspace:
 $$\beta\in \fh^*_\RR \quad \forall \beta \in \Phi.$$