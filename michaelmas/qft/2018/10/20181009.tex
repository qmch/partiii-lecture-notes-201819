Last time, we used Noether's theorem to find the stress-energy tensor
\begin{equation}
T^{\mu\nu}=\P{\cL}{(\p_\mu\phi)}\p^\nu \phi - \eta^{\mu\nu} \cL.
\end{equation}
To better understand this object, we might ask: what is $T^{\mu\nu}$ for free scalar field theory? Recall the Lagrangian for this theory is
\begin{equation}
\cL=\frac{1}{2}\p_\mu \phi \p^\mu \phi - \frac{1}{2} m^2 \phi^2.
\end{equation}
Then by explicit computation, the stress-energy tensor is
$$T^{\mu\nu}=\p^\mu \phi \p^\nu \phi - \eta^{\mu\nu} \cL.$$
The energy is given by 
$$E=\int d^3 x \left[\frac{1}{2} \dot \phi^2 +\frac{1}{2}(\grad \phi)^2 +\frac{1}{2}m^2 \phi^2 \right]$$
(from integrating the $T^{00}$ component) and the conserved momentum components are (from $T^{0i}$)
$$p^i= \int d^3x \dot \phi (\p^i \phi).$$
Note that the original Lagrangian terms don't show up here, since $\eta^{\mu\nu}$ is diagonal.

We'll note that $T^{\mu\nu}$ for this theory is symmetric in $\mu,\nu$, but a priori it doesn't have to be. If $T^{\mu\nu}$ is not symmetric initially, we can massage it into a symmetric form by adding $\p_\rho \Gamma^{\rho\mu\nu}$ where $\Gamma^{\mu\rho\nu}=-\Gamma^{\rho\mu\nu}$ (antisymmetric in the first two indices). Then $\p_\mu \left(\p_\rho \Gamma^{\rho\mu\nu}\right)=0$, which means that adding this term will not affect the conservation of $T^{\mu\nu}$. This is sufficient to attempt questions 1-6 of the first examples sheet.

\subsection*{Canonical quantization} Here, we'll follow Dirac's lead and attempt to quantize our field theories. Recall that the Hamiltonian formalism also accommodates field theories (as well as our garden-variety QM). 
\begin{defn}
We define the \term{conjugate momentum}
$$\pi(x)\equiv \P{\cL}{\dot \phi}$$
where a $\cdot$ denotes a time derivative $d/dt$, and the \term{Hamiltonian density} corresponding to a Lagrangian $\cL$ is then
$$\cH \equiv \pi(x) \dot \phi(x) - \cL(x).$$
As in classical mechanics, we eliminate the time derivative $\dot\phi$ in favor of the conjugate momentum $\pi$ everywhere in $\cH$.
\end{defn}

\begin{exm}
For $\cL=\frac{1}{2}\dot \phi^2 -\frac{1}{2}(\grad \phi)^2 - V(\phi)$ (and writing in terms of $\pi(x)=\dot \phi(x)$) we get
\begin{align*}
\cH&=(\pi)(\dot \phi)-\left(\frac{1}{2}\dot \phi^2 -\frac{1}{2}(\grad \phi)^2 - V(\phi)\right)\\
&=\frac{1}{2}\pi^2 +\frac{1}{2}(\grad \phi)^2+ V(\phi).
\end{align*}
The Hamiltonian is just the integral of the Hamiltonian density: $H=\int d^3 x \cH$. Hamilton's equations then yield the equations of motion:
$$\dot \phi = \P{H}{\pi}, \dot \pi = -\P{H}{\phi}.$$
Working these out explicitly for the free theory will give us back the Klein-Gordon equation. Note that $H$ agrees with the total field energy $E$ that we computed above.
\end{exm}

There's a slight complication in working in the Hamiltonian formalism-- because $t$ is special in our equations, the theory is not manifestly Lorentz invariant (compare to the $\p_\mu$s and variations with respect to $\delta \p_\mu \phi$ in the Lagrangian formalism). Our original theory was Lorentz invariant, so our rewritten theory is still Lorentz invariant-- it's just not immediately obvious from how we've written it.

Now let's recall that in quantum mechanics, canonical quantization takes the (classical) coordinates $q_a$ and momenta $p_a$ and promotes them to (quantum) operators. We also replace the Poisson bracket $\set{,}$ with commutators $[,]$. In QM, we had
$$[q_a,p^b]= i \delta_a^b,$$
working in units where $\hbar=1$. We'll do the same for our fields $\phi_a$ and the conjugate momenta $\pi_b$.

\begin{defn}
A \term{quantum field} is an operator-valued function of space $\phi_a(\vec x)$ obeying the commutation relations
\begin{align}
    [\phi_a(\vec{x}),\phi_b(\vec{y})] &= 0 \\
    {[\pi_a(\vec{x}),\pi_b (\vec{y})]} &= 0\\
    {[\phi_a(\vec{x}),\pi^b (\vec{y})]} &= i\delta^3(\vec{x}-\vec{y}) \delta_a^b.
\end{align}
The subscript $a$ labels which field we are talking about, and the point $\vec x$ denotes where in space we are looking.
\end{defn}

It's no coincidence that these precisely replicate the commutation relations of the operators $\hat x$ and $\hat p$ in ordinary quantum mechanics, except that now we have an additional label $\vec x$ on the fields. If you like, quantum mechanics is just a $0+1$-dimensional QFT-- there are no spatial labels to keep track of, only coordinates and momenta $q_a,p^a$. Note that $\phi_a(x), \pi^b(x)$ don't depend on $t$, since we are in the Schr\"odinger picture. All the $t$ dependence sits in the states which evolve by the usual time-dependent Schr\"odinger equation
$$i\frac{d}{dt} \ket{\psi}=H\ket{\psi}.$$
We have an infinite number of degrees of freedom, at least one for each point $x$ in space. For some theories (free theories), different solutions $\phi$ can be added together and will evolve independently-- free field theories have $L$ quadratic in $\phi_a$ (plus derivatives thereof), which implies linear equations of motion.

We saw that the simplest free theory leads to the classical Klein-Gordon equation for a real scalar field $\phi(\vec{x},t)$, i.e. $\p_\mu \p^\mu \phi+m^2\phi=0$. To see explicitly why this is a free theory, take the Fourier transform of $\phi(\vec x,t)$ to write the equations of motion in momentum space:
$$\phi(\vec{x},t)=\int \frac{d^3 p}{(2\pi)^3} e^{i \vec{p}\cdot \vec{x}}\phi (\vec{p},t).$$
Then we get the equation of motion
$$\left[\frac{\p^2}{\p t^2}+(|\vec{p}|^2+m^2)\right] \phi(\vec{p},t)=0.$$
We see that the solution is a harmonic oscillator with frequency $\omega_{\vec p} = \sqrt{\vec{p}^2 +m^2}$, so the general solution is a superposition of simple harmonic oscillators each vibrating at different frequencies $\omega_{\vec{p}}$. To quantize our field $\phi(\vec{x},t)$, we have to quantize these harmonic oscillators.

\subsection*{Review of 1D harmonic oscillators} Recall that the Hamiltonian for the simple harmonic oscillator is
$$H=\frac{1}{2} p^2 +\frac{1}{2} \omega^2 q^2,$$
subject to the canonical commutation condition $$[q,p]=i,$$ 
where $p$ and $q$ are the momentum and position operators as usual. It's certainly possible to solve this system by the series method, but the algebraic method is much more elegant by far and will generalize better. Our approach is as follows-- we'd like to factor the Hamiltonian (since if $p$ and $q$ were classical quantities we could just write it as $\frac{1}{2}(p+i\omega q)(p-i\omega q)$, for instance) but we know that this doesn't quite work because $p$ and $q$ do not commute. Therefore, we define the following operators:
\begin{itemize}
\item The creation or raising operator, $a^\dagger \equiv -\frac{i}{\sqrt{2\omega}} p +\sqrt{\frac{\omega}{2}}q$
\item The annihilation or lowering operator, $a \equiv +\frac{i}{\sqrt{2\omega}} p +\sqrt{\frac{\omega}{2}}q$.
\end{itemize}
Note that we can equivalently solve for $p$ and $q$ in terms of $a$ and $a^\dagger$: $q=\frac{1}{\sqrt{2\omega}}(a+a^\dagger)$ and $p=-i \sqrt{\frac{\omega}{2}}(a-a^\dagger)$. Substituting $p$ and $q$ into the quantization condition yields the commutator of $a,a^\dagger$,
$$[a,a^\dagger]=1.$$
We'll then factorize the Hamiltonian into $a$ and $a^\dagger$, picking up an extra term from the commutation relation of $p$ and $q$-- a little more algebra allows us to rewrite the Hamiltonian as
$$H=\frac{1}{2}\omega (a a^\dagger+ a^\dagger a)=\omega \left(a^\dagger a +\frac{1}{2}\right).$$
Computing the commutators $[H,a]$ and $[H,a^\dagger]$ reveals that
$$[H,a^\dagger]=\omega a^\dagger, [H,a]=-\omega a,$$
which tells us that the operators $a,a^\dagger$ take us between energy eigenstates.\footnote{Explicitly, consider an eigenstate $\ket{E}$ with energy $E$. Then $H a^\dagger \ket{E}=(a^\dagger H +\omega a^\dagger)\ket{E} =(E+\omega)a^\dagger \ket{E}$, so $a^\dagger \ket{E}$ is an eigenstate of $H$ with energy $E+\omega.$ The computation for $a$ is similar.} More specifically, they take us up and down a ladder of equally spaced energy eigenstates so that if we have one eigenstate with energy $E$, then we can reach a whole set of eigenstates with energy $\ldots E+2\omega, E+\omega, E, E-\omega, E-2\omega, \ldots$.

If we further postulate that the energy is bounded from below, this implies the existence of a ground state $\ket{0}$ such that the lowering operator acting on $\ket{0}$ kills the state: $a\ket{0}= 0$.\footnote{As a fun aside, the theory of angular momentum is similar, except that there for angular momentum, there is a \emph{maximum} eigenvalue as well. In fact, angular momentum is a special example of a representation of the $SU(2)$ Lie algebra-- this same structure of a ladder or lattice of states is everywhere in representation theory. See the notes for \emph{Symmetries} for more details.} In our original Hamiltonian, this ground state has energy given by
$$H \ket{0}=\omega \left(a^\dagger a +\frac{1}{2}\right)\ket{0} = \frac{\omega}{2}\ket{0},$$ so the ground state energy (or \term{zero point energy}) of the system is $\omega/2$. For our quantum theory it's really differences in energy which matter more than their absolute values,\footnote{Remark: gravity is different! Gravity couples directly to energy, not to differences in energy. But in a simple theory like the 1D harmonic oscillator, all we care about is the spacing of the energy levels.} so we could have just as easily written an equivalent Hamiltonian $H=\omega a^\dagger a$ and set the ground state energy to $0$.

We only need one state to construct our full ladder of energy eigenstates, and we can do so by passing our creation and annihilation operators back to $q$-space (real coordinates) and further writing $p=i \P{}{q}$. If we plug these back into the Hamiltonian, having set $H\ket{0}=0$, we can then treat this constraint as a differential equation to be solved for the ground state $\ket{0}$ and find that it is a Gaussian in $q$ with some appropriate variance and normalization. Then we simply need to apply $a^\dagger$ repeatedly to get all the other states, labeling them as $\ket{n}\equiv(a^\dagger)^n \ket{0}$ with $H \ket{n}=n\omega \ket{n}$. (Here we've disregarded normalization, but it's easy enough to add some scaling factor in the definition of $\ket{n}$ so that $\braket{n}{m}=\delta_{nm}$.)

That's about all there is to the quantum harmonic oscillator! From the original Hamiltonian, we have recovered the quantized energy levels and defined operators $a$ and $a^\dagger$ to move between them. Next time, we'll repeat the same procedure with quantum fields.