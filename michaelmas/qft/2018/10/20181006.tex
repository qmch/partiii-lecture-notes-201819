Last time, we derived the Euler-Lagrange equations for Lagrangian densities:
\begin{equation}
\p_\mu \P{\mathcal{L}}{(\p_\mu \phi_a)}-\P{\mathcal{L}}{\phi_a}=0.
\end{equation}
Today, we'll look at some more simple Lagrangians. We'll introduce Noether's theorem as it applies to fields and also derive the energy-momentum tensor in a field theory context.

\begin{exm}
Consider the Maxwell Lagrangian,
\begin{equation}
\mathcal{L}=-\frac{1}{2}(\p_\mu A_\nu)(\p^\mu A^\nu)+\frac{1}{2}(\p_\mu A^\mu)^2.
\end{equation}
Plugging into the E-L equations, we find that $\P{\cL}{A_\nu}=0$ and
\begin{equation}
\P{\cL}{(\p_\mu A_\nu)}=\p^\mu A^\nu +\eta^{\mu\nu}\p_\rho A^\rho.
\end{equation}
Thus E-L tells us that
\begin{equation}
0=-\p^2 A^\nu+\p^\nu (\p_\rho A^\rho)=-\p_\mu(\p^\mu A^\nu-\p^\nu A^\mu).
\end{equation}
Defining the field strength tensor $F^{\mu\nu}=\p^\mu A^\nu-\p^\nu A^\mu$, we can write the E-L equation for Maxwell as the simple
$$0=\p_\mu F^{\mu\nu},$$
which written explicitly is equivalent to Maxwell's equations in vacuum (we'll revisit this when we do QED).
\end{exm}

The Lagrangians we'll consider here and afterwards are all \term{local}-- in other words, there are no couplings $\phi(\vec{x},t)\phi(\vec{y},t)$ with $\vec{x}\neq \vec{y}$. There's no reason a priori that our Lagrangians have to take this form, but all physical Lagrangians seem to do so.

\subsection*{Lorentz invariance} Consider the Lorentz transformation on a scalar field $\phi(x)\equiv \phi (x^\mu)$. The coordinates $x$ transform as $x'=\Lambda^{-1} x$ with $\Lambda^\mu{}_\sigma \eta^{\sigma\tau}\Lambda^\nu{}_\tau = \eta^{\mu\nu}$. Under $\Lambda,$ our field transforms as $\phi\to \phi'$ where $\phi'(x)=\phi(x')$. Recall that Lorentz transformations generically include boosts as well as rotations in $\RR^3$. As we've discussed in Symmetries, Fields and Particles, Lorentz transformations form a Lie group ($O(3,1)$, or specifically the proper orthochronous Lorentz group) under matrix multipication. They have a representation given on the fields (i.e. a mapping to a set of transformations on the fields which respects the group multiplication law).

For a scalar field, this is $\phi(x)\to \phi(\Lambda^{-1}x)$ (an active transformation). We could have also used a passive transformation where we re-label spacetime points: $\phi(x)\to \phi(\Lambda x)$. It doesn't matter too much-- since Lorentz transformations form a group, if $\Lambda$ is a Lorentz transformation, so is $\Lambda^{-1}$. In addition, most of our theories will be well-behaved and Lorentz invariant.

\begin{defn}
\term{Lorentz invariant} theories are ones where the action $S$ is unchanged by Lorentz transformations.
\end{defn}

\begin{exm}
Consider the action given by
$$S=\int d^4x \left[\frac{1}{2} \p_\mu \phi \p^\mu \phi-U(\phi)\right],$$
where $U(\phi)$ is some potential density. $U\to U'(x) \equiv U(\phi'(x))= U(x')$ means that $U$ is a scalar field (check this!) and we see that
$$\p_\mu \phi' =\P{}{{x^\mu}}\phi(x')=\P{{{x'}^\sigma}}{{x^\mu}} \p_\sigma' \phi(x')= (\Lambda^{-1})^\sigma{}_\mu \p_\sigma' \phi(x')$$
where $\p_\sigma' \equiv \P{}{{{x'}^\sigma}}$.
Thus the kinetic term transforms as
$$\cL_{kin} \to \cL_{kin}'=\eta^{\mu\nu}\p_\mu \phi' \p_\nu \phi' =\eta^{\mu\nu}(\Lambda^{-1})^\sigma{}_\mu (\Lambda^{-1})^\tau{}_\nu \p_\sigma' \phi(x') \p_\tau' \phi(x')=\eta^{\sigma\tau} \p_\sigma' \phi(x')\p_\tau' \phi(x') = L_{kin}(x).$$

Thus we see that the action overall transforms as
$$S\to S' = \int d^4 x \cL(x')=\int d^4 x \cL(\Lambda^{-1}x).$$
Under a change of variables $u \equiv \Lambda^{-1} x$, we see that $\det(\Lambda^{-1})=1$ (from group theory) so the volume element is the same, $d^4y=d^4x$ and therefore
$$S'=\int d^4 y\, \cL(y)=S.$$
We conclude that $S$ is invariant under Lorentz transformations.
\end{exm}

We also remark that under a LT, a vector field $A_\mu$ transforms like $\p_\mu \phi$, so $$A_\mu'(x) = (\Lambda^{-1})^\sigma{}_\mu A_\sigma (\Lambda^{-1}x).$$
This is enough to attempt Q1 from example sheet 1.\footnote{Copied here for quick reference: Show directly that if $\phi(x)$ satisfies the Klein-Gordon equation, then $\phi(\Lambda^{-1} x)$ also satisfies this equation for any Lorentz transformation $\Lambda.$}

\begin{thm}
Every continuous symmetry of $\cL$ gives rise to a current $J^\mu$ which is conserved, $\p_\mu j^\mu=0$. Each $j^\mu$ has a conserved charge $Q=\int_{\RR^3} j^0 d^3x$.
\end{thm}
Given that the current is conserved, it's easy to show that the charge is conserved, since $\frac{dQ}{dt}=\int_{\RR^3} d^3x \p_0 j^0  = -\int_{\RR^3} d^3 x \div {\vec{j}} =0$ by the divergence theorem, assuming $|\vec{j}|\to 0$ as $|\vec{x}|\to \infty$.

Let us define an infinitesimal variation of a field $\phi$,
$\phi(x)\to \phi'(x)=\phi(x)+\alpha \Delta \phi(x)$
with $\alpha$ an infinitesimal change. If $S$ is invariant, we call this a \term{symmetry} of the theory.

Since $S$ is invariant up to adding a total 4-divergence (a total derivative $\p_\mu$) to the Lagrangian, our symmetry doesn't affect the Euler-Lagrange equations. $\cL$ transforms as
\begin{equation}\label{lagrangeinfinitesimal}
    \cL(x)\to \cL(x)+\alpha \p_\mu X^\mu(x),
\end{equation}
and expanding to leading order in $\alpha$ we have
\begin{equation}
    \cL\to \cL(x)+\alpha \P{\cL}{\phi}\Delta \phi +\alpha \P{\cL}{(\p_\mu\phi)}\p_\mu(\Delta\phi)+O(\alpha^2).
\end{equation}
We can rewrite this in terms of a total derivative $\p_\mu\left(\P{\cL}{(\p_\mu\phi)}\Delta \phi\right)$
so that
\begin{equation}
\cL'=\cL(x)+\alpha \p_\mu\left(\P{\cL}{(\p_\mu\phi)}\Delta \phi\right) +\alpha \left(\P{\cL}{\phi}-\p_\mu \P{\cL}{(\p_\mu\phi)}\right)\Delta \phi.
\end{equation}
By Euler-Lagrange, the second term in parentheses vanishes, so we identify the first term in parentheses as none other than $\alpha \p_\mu X^\mu(x)$ from Eqn. \ref{lagrangeinfinitesimal} (in other words, $\p_\mu\left(\P{L}{(\p_\mu\phi)}\Delta \phi\right) =\p_\mu X^\mu$) and recognize 
\begin{equation}
j^\mu\equiv\P{L}{(\p_\mu \phi)}\Delta \phi -X^\mu
\end{equation} as our conserved current (such that $\p_\mu j^\mu =0$).

\begin{exm}
Take a complex scalar field $$\psi(x)=\frac{1}{\sqrt{2}}(\phi_1(x)+i\phi_2(x)).$$ We can then treat $\psi, \psi^*$ as independent variables and write a Lagrangian
$$L=\p_\mu \psi^* \p^\mu \psi - V(|\psi|^2).$$
Then we observe that under $\psi\to e^{i\beta}\psi, \psi^* \to e^{-i\beta}\psi^*,$ the Lagrangian is invariant. The differential changes are $\Delta \psi = i \psi$ (think of expanding $\psi\ \to e^{i\beta}\psi$ to leading order) and similarly $\Delta \psi^*=-i\psi^*$ (here we find that $X^\mu=0$).

We add the currents from $\psi, \psi^*$ to find
$$j^\mu = i\set{ \psi \p_\mu \psi^* - \psi^* \p_\mu \psi}.$$
\end{exm}
This is enough to do questions 2 and 3 on the example sheet.
\begin{exm}
Under infinitesimal translation $x^\mu \to x^\mu -\alpha \epsilon^\mu$, we have $\phi(x)\to \phi(x)+\alpha \epsilon^\mu \p_\mu \phi(x)$ by Taylor expansion (similar for $\p_\mu\phi$). If the Lagrangian doesn't depend explicitly on $x$, then $\cL(x)\to \cL(x) +\alpha \epsilon^\mu \p_\mu \cL(x)$.

Rewriting to match the form $\cL+\a \p_\mu X^\mu$, we see that our new Lagrangian takes the form
$L(x)+\alpha \epsilon^\nu \p_\mu (\delta^\mu_\nu L)$. We get one conserved current for each component of $\epsilon^\nu$, so that
$$(j^\mu)_\nu = \P{\cL}{(\p_\mu \phi)}\p_\nu \phi - \delta^\mu_\nu \cL$$ with $\p_\mu(j^\mu)_\nu=0$.
We write this as $j^\mu{}_\nu \equiv T^\mu{}_\nu$, the energy-momentum tensor. 

\begin{defn}
The \term{energy-momentum tensor} (sometimes \term{stress-energy tensor}) is the conserved current corresponding to translations in time and space. It takes the form 
$$T^{\mu\nu} \equiv \P{\cL}{(\p_\mu \phi)}\p^\nu \phi - \eta^{\mu\nu} \cL,$$
where we have raised an index with the Minkowski metric as is conventional. The conserved charges from integrating $\int d^3x T^{0\nu}$ end up being the total energy $E=\int d^3x T^{00}$ and the three components of momentum $P^i=\int d^3x T^{0i}$.\footnote{The definition of the energy-momentum tensor here is slightly different from the one used in general relativity. Here, we have used time and space translations to derive $T^{\mu\nu}$, but in general relativity, we use variations of the metric $g^{\mu\nu}$ instead. The benefit of the GR definition is that the resulting tensor is always symmetric, whereas the $T^{\mu\nu}$ from spacetime translations is not guaranteed to be symmetric. We'll see an example of this in the example sheets, but the $T^{\mu\nu}$ defined by spacetime translations can always be \emph{made} symmetric by defining the \href{https://en.wikipedia.org/wiki/Belinfante\%E2\%80\%93Rosenfeld_stress\%E2\%80\%93energy_tensor}{Belinfante-Rosenfeld tensor}. The construction isn't anything too special, but relativists insist that variations of the action with respect to the metric is the correct way to define the energy-momentum tensor.}
\end{defn}
\end{exm}