Today we'll resolve some lingering questions about string theory in curved spacetime, starting with remarks about $\alpha' \sim l_s^2$. We'll think about $\alpha'$ corrections to general relativity, for instance.

Consider modifying the Minkowski metric to a general metric,
\begin{equation}
    \eta_{\mu\nu}\to g_{\mu\nu}(X).
\end{equation}
In general there could be some other fields in our theory like a B-field $B_{\mu\nu}(X)$ and a dilaton $\phi(X)$. Hence the action of our theory gets contributions from all of these,
\begin{equation}
    S[X,h]=S_p[X,h]+S_B[X,h]+S_d[X,h]
\end{equation}
where $p$ indicates the Polyakov action,
\begin{equation}
    S_p=-\frac{1}{4\pi \alpha'} \int_\Sigma d^2\sigma \sqrt{h} h^{ab} g_{\mu\nu}(X) \p_a X^\mu \p_b X^\nu.
\end{equation}
Since the components $g_{\mu\nu}$ now depend explicitly on $X$, this action and theory are in principle highly nonlinear.

To have a chance at solving this, we expand $X^\mu=X_0^\mu+ \eta^\mu$, where $X_0^\mu$ is a classical solution and $\eta^\mu$ is a quantum correction. For instance,
\begin{equation}
    g_{\mu\nu}(X)=G_{\mu\nu}(X_0^\mu)+\frac{1}{3} R_{\mu\lambda\sigma\nu}(X_0) \eta^\lambda \eta^\sigma + \ldots
\end{equation}
where one could show this e.g. using Riemann normal coordinates. Hence the Polyakov action takes the form
\begin{equation}
    S_p[X+\eta] = S_p[X_0] -\frac{1}{4\pi \alpha'} \int_\Sigma d^2\sigma \sqrt{h} h^{ab} \delta_{ij} \nabla_a \eta^i \nabla_b \eta^j -\frac{1}{4\pi\alpha'} \int_\Sigma d^n \sigma \sqrt{h} h^{ab} R_{\mu i j \nu}(X_0) \p_a X^\mu_0 \p^a X^\nu_0 \eta^i \eta^j + \ldots
\end{equation}
with lots of other corrections to higher powers in $\eta$. This first term looks rather like a propagator, while the second is some sort of four-point vertex.
%diagram

Exact solutions are rare, unless we are lucky or clever.%
    \footnote{``In this business, we're more often lucky than clever. Maybe that will be different in the future-- that's up to you lot.'' --R.A. Reid-Edwards}
Hence we must resort to perturbation theory-- worldsheet perturbation theory.

In flat spacetime, we had some nice properties of the stress tensor:
\begin{equation}
    T_{ab} = 0,\quad T_{++}=T_{--}=0,\quad T_{+-}=0 = \Tr(T_{ab}).
\end{equation}
Is it still true that $\avg{T_{+-}}=0$ when we do perturbation theory? Let us start with the conservation law,
\begin{equation}
    \nabla_a \avg{T_{ab}}=0,
\end{equation}
so that
\begin{equation}
    \nabla^+ \avg{T_{++}} + \nabla^- \avg{T_{-+}}=0.
\end{equation}
Switching to momentum space on $\Sigma$ with the momenta $q_-,q_+$, we have
\begin{equation}
    q_-\avg{T_{++}} + q_+ \avg{T_{-+}}=0.
\end{equation}
The computations are not too illuminating, but we'll outline the proof here. We're interested in
\begin{equation}
    \avg{T_{++}} = \int \cD \eta\, T_{++} e^{-S_p[\eta]}.
\end{equation}
To perform perturbation theory, we'll need to expand in something dimensionless, namely $\alpha'$ divided by some length scale squared. The relevant length scale will be given by the curvature of the background spacetime. That is, we expect these corrections to be valid when $\alpha'$ is small compared to the curvature of the background.

Contributions to $\avg{T_{++}}$ include some loop diagrams like
\begin{equation}
    \avg{\p_+ \eta^i \p_+ \eta_i \int_\Sigma d^2 \sigma' R_{\mu i j\nu} \p_a X^\mu_0(\sigma') \p^a X^\nu_0 (\sigma') \eta^i(\sigma') \eta^j(\sigma')}.
\end{equation}
We can rewrite this as
\begin{equation}
    \avg{T_{++}}=-\frac{1}{4} \frac{q_+}{q_-} R_{\mu i j\nu} \p_a X^\mu_0 \p^a X^\nu_0 \eta^{ij}
\end{equation}
where $\eta^{ij}$ is just the Minkowski metric (unrelated to our perturbations).
Similarly,
\begin{equation}
    \avg{T_{-+}} = \frac{1}{4} R_{\mu\nu} \p_a X_0^\mu \p_b X^\nu h^{ab}.
\end{equation}
Critically, this is perturbation theory on the worldsheet, not in terms of string scattering. According to this calculation, there's now no guarantee that the off-diagonal elements vanish, though this is okay if $R_{\mu\nu}(X_0)=0$.

Moreover, we could get contributions from the B-field and the dilaton:
\begin{gather}
    S_B=-\frac{i}{4\pi \alpha'} \int_\Sigma d^2 \sigma \sqrt{h} \epsilon^{ab} \p_a X^\mu \p_b X^\nu B_{\mu\nu}(X)\\
    S_\phi = -\frac{1}{4\pi} \int_\Sigma d^2 \sigma \sqrt{h} R_\Sigma \phi(X),
\end{gather}
where $\epsilon^{ab}$ is the totally antisymmetric rank two tensor. Taking the entire action to be the sum of a Polyakov action and the B-field and dilaton contributions, we get more Feynman rules and more corrections. If we calculate these corrections, we find that
\begin{align*}
    \avg{T_{-+}} ={}&\frac{1}{4} \paren{R_{\mu\nu}-\frac{1}{4} H_{\mu\nu}^2 + 2\nabla_\mu \nabla_nu \phi} \p_a X_0^\mu \p_b X_0^\nu h^{ab}\\
    &+ \frac{1}{4} \paren{\nabla^\lambda H_{\lambda\ mu\nu}-2 \nabla^\lambda \phi H_{\lambda \mu\nu} } \p_a X_0^\mu \p_b X_0^\nu \epsilon^{ab}\\
    &+ \frac{1}{\alpha'} \paren{\frac{D}{2} +\frac{\alpha'}{2} \paren{-R+\frac{H^2}{12} +4(\nabla \phi)^2 -4\nabla^2 \phi}} \p_a X^\mu_0 \p_b X_{0\mu}  h^{ab} +O(\alpha'{}^2).
\end{align*}
These lines represent contributions from the metric, the B-field, and the dilaton respectively. We haven't included the ghosts-- as it turns out, they are largely indifferent to whether we're working in flat or curved spacetime, and will just modify the $D/2$ term in the dilaton contribution (the third line) with a $D-26$. Also, note that we have to work to two-loop order in the dilaton because it does not come with a $1/\alpha'$ like the other actions.

For these to be classically compatible, we may think of the vanishing of the trace of the stress tensor as giving us three equations of motion for $g_{\mu\nu}(X_0),B_{\mu\nu}(X_0),\phi(X_0)$. These can be derived from the action
\begin{equation}
    S=\int d^{26}x \sqrt{g} e^{-2\phi}\paren{R+ 4(\nabla \phi)^2 -\frac{1}{12} H^2}.
\end{equation}
In fact, the equations of motion which emerge are precisely the equations of general relativity. This is remarkable. From a quantum consistency condition, we have derived Einstein's equations.

Can we go further? It takes more loop corrections, but it can be done. By performing the two-loop calculations for $S_p[X]$, we find that
\begin{equation}
    R_{\mu\nu}+\frac{\alpha'}{2} R_{\mu k \lambda \sigma} R_\nu{}^{k\lambda \sigma}+\ldots =0,
\end{equation}
which represent higher-order ``corrections'' to general relativity.

The problem is then to find a background metric, B-field, and dilaton that solve the equations of motion to \emph{all orders} in loop calculations, and then do perturbation theory. This is really hard-- it requires us to perform high-order loop calculations just to write down the equations of motion. We don't have a general scheme for solving these equations, so we will have to be clever, lucky, or both.

\subsection*{T-duality}
There is however one trick we have to solve the equations. Consider a spacetime
\begin{equation}
    \cM_{26}=\RR_{1,24} \times S^1,
\end{equation}
exact to all orders in $\alpha'$ with $B,\phi$ constant. Let the 25th (i.e. the periodic $S^1$) coordinate satisfy
\begin{equation}
    X^{25}(\sigma+2\pi \tau)= X^{26}(\sigma,\tau)+2\pi Rm, m \in \ZZ.
\end{equation}
Thus
\begin{equation}
    X^{25}(\sigma,\tau)=x^{25}+\alpha' p^{25} \tau + mR \sigma +\sum_{n\neq 0}(\text{oscillations}).
\end{equation}
Also, we'll take $p^{25}=n/R, n\in \ZZ$, so that the momentum is quantized by periodicity. The mass spectrum of such a theory is
\begin{equation}
    \alpha' M^2 = \alpha'\frac{n^2}{R^2} +\frac{1}{\alpha'} m^2 R^2 +(N+\bar N-2).
\end{equation}
Hence there is energy in the oscillations (the last term, where $N,\bar N$ are oscillation number operators) and also the winding number $m$.

We notice there's a symmetry of this spectrum. If we interchange
\begin{gather*}
    m\leftrightarrow n,\\
    \frac{\alpha'}{R^2} \leftrightarrow \frac{R^2}{\alpha'},
\end{gather*}
we see that the spectrum is left unchanged. In fact, it goes further-- this is actually a symmetry of the entire theory. That is, by exchanging the momentum number and the winding number, and then inverting the radius, we get an equivalent description.

These statements generalize to other spacetimes with interesting geometries and topologies-- not just tori and handlebodies but Calabi-Yau manifolds. This tells us that the string sees a structure not present in standard Riemannian geometry. In some sense, this is our first hint of a structure of spacetime beyond Einstein's geometric description, and it has led to much interesting research into what a (or the) theory of quantum gravity might look like.