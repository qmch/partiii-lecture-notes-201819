Today we'll begin our discussion of scattering amplitudes in string theory, i.e. the S-matrix.

\subsection*{The big idea}
Recall that we found it useful to conformally map our worldsheet cylinder onto the complex plane, or equivalently the Riemann sphere ($\CC \cup \set{\infty}$) with two punctures, using the map $z=e^{\tau + i\sigma}$. What if there are some initial and final states $\ket{\phi_i},\ket{\phi_f}$ in the picture?

We can encode the initial and final states in the Riemann sphere picture by inserting operators $V_i,V_f$ at the punctures, where e.g.
\begin{equation}
    \ket{\phi_i}=\lim_{z\to 0} V_i(z) \ket{0}.
\end{equation}
But what if we want to discuss scattering? We could think of interactions between strings as described by worldsheets with many boundaries. In the same way, we will assume that $\exists$ a (conformal) map between e.g. a state with two initial strings merging and separating (see diagram) to a sphere with four punctures. 
%diagram
We will also have loop diagrams, and these can be mapped to tori with punctures.%
    \footnote{We can definitely construct this from the ``tree-level'' interactions by ``gluing'' the punctured Riemann spheres together at the punctures.}
%diagram    

Note that there have always been two theories in the game-- the physics on our worldsheet, and the bigger spacetime it was embedded in. In order to properly discuss scattering amplitudes, we will need a few preliminaries.

\subsection*{Scattering preliminaries}
What constraints on the ghosts are required for the limit
\begin{equation}
    \ket{\phi}=\lim_{z\to 0} \phi(z) \ket{0}
\end{equation}
to exist? Suppose $\phi(z)$ is of weight $(h,0)$ (a chiral field), such that
\begin{equation}
    \phi(z) = \sum_n \phi_n z^{-n-h}.
\end{equation}
Then the limit we want to evaluate is
\begin{equation}
    \lim_{z\to 0} \sum_n \phi z^{-n -h} \ket{0}.
\end{equation}
Notice that for $-n-h >0$, the terms go to zero as $z\to 0$, but for $-n-h <0$ these terms will generically blow up as $z\to 0$.

We can get a sensible limit if we require that
\begin{equation}
    \phi_n \ket{0} =0,\quad n > -h,
\end{equation}
and then
\begin{equation}
    \ket{\phi}=\phi_{-h} \ket{0},
\end{equation}
the only mode that doesn't vanish.

For the ghosts, $b$ has weight $h=2$ and $c$ has weight $h=-1$. This means that
\begin{equation}
    c_0\ket{0} \neq 0, c_1 \ket{0} \neq 0
\end{equation}
and
\begin{equation}
    \bra{0} c_{-1} c_0 c_{+1}\ket{0} \neq 0.
\end{equation}
There's some freedom in how we choose to normalize the ghost vacuum.
We can choose
\begin{equation}
    \bra{0} c_{-1} c_0 c_{+1} \ket{0} =1
\end{equation}
%
One can then show as an exercise that the expectation of the $c$-ghosts at three points (e.g. by a mode expansion) that
\begin{equation}
    \bra{0} c(z_1) c(z_2) c(z_3) \ket{0} = (z_1-z_2)(z_2-z_3)(z_3-z_1).
\end{equation}
\subsection*{The dilaton and the string coupling}
There's also an interesting point to be made about the dilaton, the scalar that popped out of our theory early on. We could consider the string as propagating in a spacetime with a background of $g_{\mu\nu}(X),B_{\mu\nu}(X),\Phi(X)$. That is, there's other stuff like background curvature and EM fields in the ambient spacetime. In that case, our worldsheet theory ought to be sensitive to this stuff.

In particular, the worldsheet metric would be modified to
\begin{equation}
    S=-\frac{1}{4\pi \alpha'} \int_\Sigma d^2\sigma \sqrt{-h} h^{ab} \p_a X^\mu \p_b X^\nu g_{\mu\nu}(X).
\end{equation}
This is hard to solve. We already needed perturbation theory just to discuss the string in a flat Minkowski background-- now we have some additional structure to perturb about. We might also pick up a factor
\begin{equation}
    S=-\frac{i}{4\pi \alpha'} \int_\Sigma d^2\sigma \sqrt{-h} \epsilon^{ab} \p_a X^\mu \p_b X^\nu B_{\mu\nu}(X)
\end{equation}
with $\epsilon^{ab}=\begin{pmatrix} 0 & -1\\ 1& 0\end{pmatrix}.$ If we like, the $B$-field is just a two-form, and its contribution to the action is simply the pullback of this two-form to the worldsheet.

There's one last thing we could do-- we could couple to the dilaton.
\begin{equation}
    S_\Phi = \frac{1}{4\pi} \int_\Sigma d^2\sigma \sqrt{-h} \Phi(X) R_\Sigma,
\end{equation}
where $R_\Sigma$ is the Ricci scalar on $\Sigma$. There are a few strange features of this-- this isn't at order $1/\alpha'$ but at order 1. The structure of this coupling also looks different than the other two, as it is diff invariant but not Weyl invariant. Moreover, if $\Phi(X)$ has a vacuum expectation value $\avg{\Phi(X)}=\Phi_0$, then our action picks up a contribution
\begin{equation}
    S_\Phi= \frac{1}{4\pi} \Phi_0 \int_\Sigma d^2\sigma \sqrt{-h} R_\Sigma =\Phi_0 \chi=\Phi_0 (2g-2)
\end{equation}
where $\chi$ is the Euler characteristic of $\Sigma$ and $g$ is the genus of the worldsheet $\Sigma$.

So in the path integral, the sum over genus is weighted by a factor of $e^{\Phi_0}$. Our path integral has the form
\begin{equation}
    Z=\sum_{g=0}^\infty\frac{e^{\Phi_0(2g-2)}}{|\text{CKG}|}\int_{\cM_g} d^st \int \cD b \cD c \cD \bar b \cD \bar c \cD X \prod_{I=1}^s (\mu_I | b)(\bar \mu_I|\bar b) e^{-S[X,b,c]}.
\end{equation}
That is, in addition to the path integrals over the fields and ghosts, we must sum over all topologies with a weight given by the dilaton VEV, where we call
\begin{equation}
    g_c \equiv e^{\Phi_0}
\end{equation}
the closed string coupling constant.
%I guess everyone knows gravitons are blue.

For suppose we have a Riemann surface of genus $g$, weighted by $g_c^{2g-2}$. Now if we have a closed string state (e.g. a graviton) being emitted and then absorbed by $\Sigma_g$, this adds one more handle to $\Sigma_g$ and hence increases the genus by $1$. Thus $g\to g+1$ adds another factor of $g_c^2$ to our expression, so we associate a factor of $g_c$ with each additional ``vertex,'' i.e. one for the emission and one for the absorption in this process.

For now, we'll assume the background metric is flat Minkowski and there's no $B$-field. But there may be a nonzero dilaton field.

\subsection*{Vertex operators}
How can we build operators that live in the BRST cohomology? Suppose we have an operator $\Phi(z,\bar z)$ which satisfies
\begin{equation}
    [Q_B,\phi(z,\bar z)]=\p(c\phi),\quad [\bar Q_B, \phi(z,\bar z)]=\bar \p(\bar c \phi).
\end{equation}
This is not chiral; it knows about both the holomorphic and antiholomorphic sectors of the theory. Then
\begin{equation}
    V_\phi = \int_\Sigma d^2z \, \phi(z,\bar z)
\end{equation}
is BRST-closed. Given one such solution, we'll get some others for free. Consider
\begin{equation}
    U(z,\bar z) =c(z) \bar c(z) \phi(z,\bar z).
\end{equation}
By linearity,
\begin{align*}
    [\cQ_B, U]=[Q_B +\bar Q_B, U] &= c\p c \bar c \phi + c\bar c \p(c \phi)+\text{barred expressions}\\
        &= c\p c \bar c \phi + c \bar c \p c\phi +\ldots\\
        &= 0.
\end{align*}
Thus $U$ is BRST-closed.%
    \footnote{The $\p\phi$ term goes away since it has a $c^2$, and the two terms we've written cancel once we anticommute $\bar c$ and $\p c$.}
We have therefore constructed two BRST-closed objects, one local and one non-local.
