\subsection*{SSB of a continuous (global) symmetry} Consider a real $N$-component scalar field $\phi=(\phi_1,\phi_2,\ldots,\phi_N)^T$, with the Lagrangian
\begin{equation}
    \cL=\frac{1}{2}(\p_\mu \phi)\cdot (\p^\mu \phi)-V(\phi),
\end{equation}
where
\begin{equation}
    V(\phi)=\frac{1}{2}m^2 \phi^2 +\frac{\lambda}{4}\phi^4, \quad \phi^2=\phi \cdot \phi, \phi^4=(\phi^2)^2, \lambda >0.
\end{equation}
This Lagrangian is invariant under global $O(N)$ symmetry, $\phi \to O \phi$ where $O^TO=I$. If $m^2>0$ we can expand about the $\phi=0$ minimum, so here we'll be interested in $m^2<0$ again.

In this case,
\begin{equation}
    V(\phi)=\frac{\lambda}{4}(\phi^2-V^2)^2
\end{equation}
up to an irrelevant constant factor, where $V^2=-m^2/\lambda >0$. This is alternately known as the ``sombrero,'' ``Mexican hat,'' or ``wine bottle potential.''%
    \footnote{Savvy readers may already know this is connected to the Higgs mechanism.}
    
The minima of $V(\phi)$ now trace out a circle (or a sphere in higher dimensions), $\phi^2=V^2$. As before, the $\phi=0$ vacuum is unstable, so we must expand about a stable vacuum in order to study the field. WLOG we'll choose $\phi_0=(0,0,\ldots,V)^T$ (so that $V(\phi_0)=0$) and study small fluctuations about this,
\begin{equation}
    \phi(x)=(\pi_1(x),\pi_2(x),\ldots,V+\sigma(x))^T
\end{equation}
where $\pi(x)$ is a real scalar field with $N-1$ components and $\sigma(x)$ has one component.

In terms of our new fields, the Lagrangian becomes
\begin{equation}
    \cL = \frac{1}{2} (\p_\mu \pi)\cdot (\p^\mu \pi)+\frac{1}{2}(\p^\mu\sigma)(\p_\mu \sigma)-V(\pi,\sigma).
\end{equation}
Our potential is now modified so that only the field $\sigma$ has a mass while the other $\pi$ fields remain massless:
\begin{equation}
    V(\pi,\sigma)=\frac{1}{2} m_\sigma^2 \sigma^2 + \lambda V(\sigma^2 +\pi^2)\sigma +\frac{\lambda}{4}(\sigma^2+\pi^2)^2.
\end{equation}
The $\sigma$ field has a mass
\begin{equation*}
    m_\sigma^2 =2\lambda V^2
\end{equation*}
as before, but the $N-1$ $\pi$ fields are massless. We also observe the that potential has a third-order term $O(\sigma^3)$, which tells us that the new Lagrangian does not have the old $\phi\to-\phi$ symmetry when expanded about a stable vacuum. The $\sigma$ field gives radial excitations in $V(\phi)$, while the $\pi$ fields are excitations in the azimuthal (angular) direction, i.e. flat directions.

Let's now generalize this analysis to a symmetry group $G$ of a Lagrangian $\cL$ which is broken to a subgroup $H\subset G$ by the vacuum. We'll generally be considering normal subgroups. That is, let us transform the field $\phi\to g\phi$ with $g\in G$ in some representation, so that $\cL(\phi)=\cL(g\phi)\forall g\in G$. Thus $g$ is a symmetry of the entire Lagrangian (and therefore the action).

Assume that $G$ is spontaneously broken, and hence the vacuum is not unique but a manifold%
    \footnote{I think we basically get this for free since we are imposing an algebraic constraint on the fields of the form $f(\phi)={}$constant, and the fields without the constraint formed a manifold ($\RR^n$, $\CC^n$, etc.).}
\begin{equation}
    \Phi_0=\set{\phi_0: V(\phi_0)=V_{\min{}}}.
\end{equation}
The invariant subgroup (or stability group) $H\subset G$ is
\begin{equation}
    H=\set{h\in G: h\phi_0 = \phi_0}.
\end{equation}
That is, choose a vacuum state $\phi_0$ and consider all the elements which leave $\phi_0$ unchanged. Different vacua are related by $\phi_0'=g\phi_0$, with $\phi_0,\phi_0'\in \Phi_0$. Note that the stability groups for different vacua are isomorphic, i.e. for $\phi_0'$ the stability group is all elements $h'=ghg^{-1}$, i.e. $H'\simeq gH g^{-1}$.

The group elements that map one vacuum to another are in the coset space $G/H$ and fall into the equivalence classes,
\begin{equation}
    g_1 \sim g_2 \text{ if }\exists \,h\in H \text{ s.t. } g_1=g_2h.
\end{equation}
That is, two elements of the group are equivalent if they are in the same left coset. Thus
\begin{equation}
    \phi_0'=g_1 \phi_0 = g_2 \phi_0 \implies g_2^{-1} g_1\in H.
\end{equation}
That is, this element is in the stabilizer, and so there is one equivalence class (coset) for each $\phi_0'\in \Phi_0: \Phi_0 \simeq G/H$. This is itself a group if $H$ is a normal subgroup. That is, if we quotient out by the stabilizer, we get just the elements which take us between vacua.

Let's now consider Lie groups (i.e. groups with manifold structure). In the context of Lie groups, we can look at infinitesimal transformations:
\begin{equation}
    g\phi=\phi+\delta \phi, \quad \delta \phi = i\alpha^a t^a \phi
\end{equation}
where $a=1,\ldots, \dim G$, $t^a$ are the generators of the Lie algebra in the representation acting on $\phi$, and $\alpha^a$ are small parameters. The invariance of the potential under $G$ now means that $V(\phi+\delta \phi)=V(\phi)$, or expanding out,
\begin{equation}
    V(\phi+\delta \phi)-V(\phi)= i\alpha^a (t^a \phi)r \P{V}{\phi_r}=0
\end{equation}
to first order, where $r=1,2,\ldots, N$ indexes over components of $\Phi$ in its representation. If $\phi_0$ is a minimum of $V$, then by looking at the second-order terms we get
\begin{equation}
    V(\phi+\delta \phi)-V(\phi)= \frac{1}{2} \delta \phi_r \frac{\p^2 V}{\p \phi_r \p \phi_s}\delta \phi_s+\ldots,
\end{equation}
where this second derivative is defined to be the ``mass matrix,''
\begin{equation}
    M_{rs}^2 \equiv \frac{1}{2} \delta \phi_r \frac{\p^2 V}{\p \phi_r \p \phi_s}\delta \phi_s.
\end{equation}