\begin{quote}
    \textit{``This [tidal force divergence] is related to something that happens in AdS/CFT. \ldots don't tell anyone I mentioned that.'' --Jorge Santos}
\end{quote}

Quick announcement-- office hours from this course will be held at 14:00 on Tuesdays. If you plan to attend the office hours, however, do send an email in advance.

Last time, we started looking at the geodesics of Schwarzschild. We found two particularly simple ones: the ingoing and outgoing radial null geodesics, with equation
\begin{equation}
    \frac{dt}{dr} = \pm \paren{1-\frac{2M}{r}
    }^{-1}, \quad r> 2M.
\end{equation}
We also defined the tortoise (Regge-Wheeler) coordinate $r_*$ such that
\begin{equation}
    dr_* =\frac{dr}{1-\frac{2M}{r}}.
\end{equation}
Thus our null geodesic equation becomes
\begin{equation}
    \frac{dt}{dr_*}=\pm 1.
\end{equation}
In tortoise coordinates, null geodesics therefore obey
\begin{equation}
    t=\pm r_* + \text{constant}.
\end{equation}
This seems kind of trivial. But now we introduce a new coordinate, the \term{ingoing Eddington-Finkelstein coordinate}, defining
\begin{equation}
    v \equiv t+ r_*.
\end{equation}
This is clearly constant on ingoing geodesics, just by looking at the previous equation. That is, $dv=dt+dr_*=0.$ Good.%
    \footnote{We could have done the same for outgoing geodesics taking $u\equiv t-r_*,$ and indeed we will do so later.} 
Now we eliminate the original coordinate time $t$ from the line element using $dt=dv-\frac{dr}{1-\frac{2M}{r}}.$ We get the new line element
\begin{equation}
    ds^2 = -\paren{1-\frac{2M}{r}} dv^2 +2dv dr +r^2 d\Omega^2_2.
\end{equation}
%black holes are type D solutions
Let's write this in matrix notation. Nothing fancy.
\begin{equation}
    g_{\mu\nu}=\begin{pmatrix}
        -1+\frac{2M}{r} & 1 & 0 & 0\\
        1 & 0 & 0 & 0\\
        0 & 0 & r^2 & 0\\
        0 & 0 & 0 & r^2\sin^2\theta
    \end{pmatrix}.
\end{equation}
We haven't done anything too extreme, just made a change of coordinates. But we see something very nice-- none of the metric components are singular at $r=2M$. In fact, the determinant of the metric is still perfectly nice at $r=2M$-- by an explicit computation, $\det g= -r^2 \sin^2 \theta$. This only vanishes at $\theta=0$, which is the regular coordinate badness%
    \footnote{AKA degeneracy. Basically, what is the value of $\phi$ at the north pole? It's not well-defined, hence our coordinates do not define an invertible map from coordinates to manifold points.}
of spherical coordinates near the poles, and at $r=0$, which may (a priori, we don't know yet) be a real problem.

So we have found some coordinates which appear to extend $r$ not just from $r>2M$ but to all $r>0$. Our metric is real and analytic so it is a nice analytic continuation of the old (bad) Schwarzschild coordinates. This is related to the problem of extendibility-- are there other metrics which cover more of the spacetime manifold which are compatible with the solution that we've found?

However, something really bad does happen as $r\to 0$. The Kretchmann scalar $R^{abcd}R_{abcd}=\frac{48 M^2}{r^6} \to \infty$ as $r\to 0$, and scalars by definition are invariant under coordinate transformations. So we cannot get rid of this by a simple redefinition of coordinates.
%This [tidal force divergence] is related to something that happens in AdS/CFT. ...don't tell anyone I mentioned that.

Moreover, let us observe that $\P{}{t}=\P{}{v}$. Thus our Killing vector field becomes
\begin{equation}
    K=\P{}{v}, \quad K^2 =-\paren{1-\frac{2M}{r}}.
\end{equation}
So our metric appears to be no longer static or stationary in these coordinates.

We now introduce the Finkelstein diagram, shown in Figure \ref{fig:reall_finkelsteindiagram}. Recall that for $r=2M$, on outgoing geodesics we have $t-r_*=$constant $\implies v= 2r+4 M \log \abs*{\frac{r}{2M}-1}+$constant. Let us draw a plot in the $t_* \equiv v-r$,$r$ plane. What we see is that the ingoing geodesics follow 45${}^\circ$ paths, while the outgoing geodesics follow some curved paths in these coordinates. %copy picture from Harvey Reall notes

\begin{figure}
    \centering
    \includegraphics[width=0.8\textwidth]{2019/01/20190124_reall_finkelsteindiagram.png}
    \caption{
    The Finkelstein diagram for the Schwarzschild spacetime. The sides of the light cones are given by lines of constant $u$ and $v$. Notice the light cones ``tip over'' as we cross the event horizon at $r=2M$, so that even ``outgoing'' radial null trajectories must proceed towards the $r=0$ singularity.
    \newline
    Image credit to Prof. Reall's  \href{http://www.damtp.cam.ac.uk/user/hsr1000/black_holes_lectures_2016.pdf}{Black Holes notes}, \textsection 2.5.
    }
    \label{fig:reall_finkelsteindiagram}
\end{figure}

What we observe is that in these coordinates, the light cones ``tip over'' at $r=2M$-- even the ``outgoing'' null geodesics are forced to proceed towards $r=0$ for $r<2M$. Now, this does not yet prove that this is a black hole. We've only examined radial geodesic trajectories, and it's not clear that adding even a little bit of angular momentum can't save us from a spaghettified death by black hole(-like object).
%

Suppose that we%
    \footnote{``Imagine that you, not me, are situated at the surface of this star. I don't want to go there.'' -- Jorge Santos
    }
sit on the surface of a collapsing star. What we can show is that the time to singularity is $\Delta t = \mathbb{T} M$, where $M=M_\odot \implies \Delta t= \SI{10e-5}{\second}.$ It's a weird fact that we can indeed cross the horizon and hit the singularity in finite time, and yet because the event horizon is a surface of infinite redshift, a far-away observer will never see us actually cross the event horizon.
%Pictures are nice, but they can be very deceiving. ...try to go to a dating site.

\subsection*{Black hole region} 
How do we define a black hole? Before we can do that, we'll need some preliminaries.
\begin{defn} 
    A vector is \term{causal} if it is null or timelike and nonzero. A curve is causal if its tangent vector is everywhere causal.
\end{defn}
\begin{defn}
    A spacetime is \term{time-orientable} if it admits a time orientation, i.e. a causal vector field $T^a$. Another causal vector field $x^a$ is then \term{future-directed} if it lies in the same light cone as $T^a$ and is past-directed otherwise (i.e. $x^a T_a \leq 0$).
\end{defn}
Note that the old coordinate time $t$ in Schwarzschild was really a bad choice inside the event horizon. This is because $\P{}{t}$ becomes spacelike inside the horizon, related to the change of sign of $g_{tt}$ for $r<2M$.
So let us take $\pm \P{}{r}$, and note that in our EF coordinates, $g_{rr}=0$. Therefore $\P{}{r}\P{}{r}=0$, meaning that we've found a vector which is null everywhere.

In fact, recall our timelike Killing vector outside, $K$. This gave us a good sense of time outside. We see now that
\begin{equation*}
    K \cdot \paren{-\P{}{r}} =-g_{vr} = -1,
\end{equation*}
where $K\equiv \P{}{v}$. Therefore we've found a timelike coordinate which is good everywhere and defines a good time orientation.

\begin{prop}
Let $x^\mu(\lambda)$ be any future-directed causal curve, i.e. one whose tangent vector is everywhere future-directed and causal. Assume $r(\lambda_0)\leq 2M$. Then $r(\lambda)\leq 2M$ for any $\lambda \geq \lambda_0.$
\end{prop}

Let us define the tangent vector $V^\mu = \frac{dx^\mu}{d\lambda}$. Since $\paren{-\P{}{r}}$ and $V^a$ are both future-directed, we have
\begin{equation}
    0\geq \paren{-\P{}{r}} V = -g_{r\mu} V^\mu = -V^v = -\frac{dv}{d\lambda} \implies \frac{dv}{d\lambda} \geq 0.
\end{equation}
Now
\begin{equation}
    V^2 \equiv V^a V_a = -\paren{1-\frac{2M}{r}}\paren{\frac{dv}{d\lambda}}^2 +2\paren{{dv}{d\lambda}} \paren{\frac{dr}{d\lambda}}+r^2 \paren{\frac{d\Omega}{d\lambda}}^2,
\end{equation}
where $\paren{\frac{d\Omega}{d\lambda}}^2 \equiv \paren{\frac{d\theta}{d\lambda}}^2 +\sin^2 \theta \paren{\frac{d\phi}{d\lambda}}^2.$
Then
\begin{equation}\label{dvdlambdacondition}
    -2\paren{\frac{dv}{d\lambda}}\paren{\frac{dr}{d\lambda}}=-V^2 +\paren{\frac{2M}{r}-1} \paren{\frac{dv}{d\lambda}}^2 + r^2 \paren{\frac{d\Omega}{d\lambda}}^2.
\end{equation}
But if $V$ is causal and $r\leq 2M,$ then the right side of \ref{dvdlambdacondition} is non-negative, so it follows that
\begin{equation}
    \frac{dv}{d\lambda}\frac{dr}{d\lambda} \leq 0.
\end{equation}
Let us assume that $\frac{dr}{d\lambda}>0$ (our curve at any point is directed towards larger $r$). Then $\frac{dv}{d\lambda}=0$, which means that by \ref{dvdlambdacondition}, $V^2=0$ and $\frac{d\Omega}{d\lambda}=0$. But then the only nonvanishing component of $V$ is $V^r=\frac{dr}{d\lambda} >0 \implies V$ is a positive multiple of $\P{}{r}$, and hence is past-directed. We have reached a contradiction.

Therefore $\frac{dr}{d\lambda}\leq 0$ if $r\leq 2M$. If $r< 2M$, the equality must be strict. If $\frac{dr}{d\lambda}=0$ then by \ref{dvdlambdacondition}, $\frac{d\Omega}{d\lambda}=\frac{dv}{d\lambda}=0 \implies V^\mu=0$. Hence if $r(\lambda_0) < 2M$, then $r(\lambda)$ is monotonically decreasing for all $\lambda \geq \lambda_0$.