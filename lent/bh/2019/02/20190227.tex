\begin{quote}
    \textit{``If you want to work in mathematical GR, this is the problem to tackle. Your supervisor will never give you this problem. But this is the one.'' --Jorge Santos}
\end{quote}

Last time, we finally defined a black hole.
%If you try to prove the singularity thm assuming a singularity, you find (in tracing the inextendible causal geodesics that there is a set which is closed but not compact since the singularity is removed from the spacetime.
Recall that the (future) horizon is
\begin{equation*}
    \cH^+ = \cM \cap \dot J^{-}(\mathcal{I}^+).
\end{equation*}
The horizons $\cH^+$ are null hypersurfaces, and the generators of $\cH^+$ cannot have future endpoints. For instance, we saw this in the spherical collapse of a star.

Today we'll show in full generality that black holes cannot bifurcate. To prove this, we'll need some definitions.
\begin{defn}
    An asymptotically flat spacetime $(\cM,g)$ is \term{strongly asymptotically predictable} if $\exists$ an open region $\bar V \subset \bar \cM$ such that $\overline{\cM \cap J (I^+)} \subset \bar V$ and $(\bar V, g)$ is globally hyperbolic.
\end{defn}
That is, if we start with a partial Cauchy surface and some initial data on the horizon, we can predict everything.

\begin{thm}
    Let $(\cM,g)$ be strongly asymptotically predictable, and let $\Sigma_1,\Sigma_2$ be Cauchy surfaces for $\bar V$ with $\Sigma_2\subset I^+(\Sigma_1)$. Let $B$ be a connected component of $\cB \cap \Sigma_1$. Then $J^+(B)\cap \Sigma_2$ is contained within a connected component of $\cB\cap \Sigma_2$.
\end{thm}
\begin{proof}
    The proof is by contradiction. Global hyperbolicity implies that a causal curve from $\Sigma_1$ will intersect $\Sigma_2$. Note that $J^+(B)\subset \cB$, so $J^+(B)\cap \Sigma_2 \subset \cB\cap \Sigma_2$. Thus the causal future of the black hole region on $\Sigma_1$ is contained within the black hole region on $\Sigma_2$.
    
    Now assume that $J^+(B)\cap \Sigma_2$ is not constrained within a single connected component of $\cB \cap \Sigma_2$. By definition, we can find disjoint open sets $O,O' \subset \Sigma_2$ such that $J^+(B) \cap \Sigma_2 \subset O\cup O'$ with $J^+(B)\cap O \neq \emptyset$ and $J^+(B)\cap O' \neq \emptyset$.
    
    Then $B \cap I^-(O)$ and $B \cap I^-(O')$ are non-empty and $B\subset I^-(O)\cup I^- (O')$. Now $p\in B$ cannot lie in both $I^-(O)$ and $I^-(O')$, for then we could divide future-directed timelike geodesics into two sets according to whether they intersect $O$ or $O'$, and hence divide future-directed timelike vectors at $p$ into two disjoint open sets, contradicting connectedness of the future light cone at $p.$
\end{proof}
%Imagine that we're in five dimensions. THen it turns out that black holes can have many topologies. One really nice solution is called a black string. Take the Schwarzschild black hole. Add dz^2. A boring flat direction. What is the topology of the horizon? The topology of Schwarzschild is S^2. Add a dz^2 and we get a cylinder.
%Go home and turn on your faucet. You get a filament of water. If you turn it off, you get droplets that keep you awake at night. It turns out the exact same thing happens for the black string.
%these droplets will eventually go down to the Planck scale, quantum effects kick in, and we get a naked singularity. 
%now I want you to forget about black holes for one minute.

\subsection*{Weak cosmic censorship}
Weak cosmic censorship is the biggest outstanding problem in mathematical general relativity. Consider the Penrose diagram for the Schwarzschild black hole with $M<0$. This spacetime has a naked singularity, a singularity that is visible from timelike infinity.. But maybe such spacetimes are unrealistic-- can we form a naked singularity through gravitational collapse?

Birkhoff's theorem tells us we can't have a spherically symmetric collapse in pure gravity because the most general spherically symmetric solution is static. So suppose we add in a scalar field. The region where the naked singularity is visible is $\overline{D^+(\Sigma)}\setminus I(D^+(\Sigma))$.
%What have we done to scri-plus? We have mutilated it!
In fact, we really shouldn't talk about the singularity itself. Beyond the Cauchy horizon, solutions become highly non-unique, so we should remove that region from the spacetime. Instead, it's more appropriate to talk about $\mathcal{I}^+$-- if $\mathcal{I}^+$ is incomplete, then we have a naked singularity.

\textbf{Conjecture.} Weak cosmic censorship (Horowitz and Geroch). \textit{Let ($\Sigma, h, K)$ be a geodesically complete, asymptotically flat initial data set. Let the matter fields obey hyperbolic equations and satisfy the dominant energy condition. Then generically the maximal development of this initial data is an asymptotically flat spacetime.}
%If you want to work in mathematical GR, this is the problem to tackle. Your supervisor will never give you this problem. But this is the one.

\subsection*{Apparent horizons}
\begin{thm}
    Let $T$ be a trapped surface in a strongly asymptotically predictable spacetime obeying the null energy condition. Then $T\subset \cB$ (the trapped surface is within the black hole region).
\end{thm}
\begin{proof}
    Assume there exists $p\in T$ such that $p\not \in \cB$. Thus $p$ is in the causal past of $\mathcal{I}^+$, i.e. $p\notin J^-(\mathcal{I}^+).$ Then there exists a causal curve from $p$ to $\mathcal{I}^+$.
    
    One can then use strong asymptotic predictability to show that this implies that $\dot J^+(T)$ must intersect $\mathcal{I}^+$, i.e. there exists a $q\in \mathcal{I}^+$ with $q\in \dot J^+(T)$. But we have seen (from two lectures ago) that $q$ lies on a null geodesic $\gamma$ from $r\in T$ that is orthogonal to $T$ and has no point conjugate to $r$ along $\gamma$. Since $T$ is trapped, the expansion of the null geodesics orthogonal to $T$ is negative at $r$, and hence $\theta\to -\infty$ within finite affine parameter along $\gamma$. Thus $\exists$ a point $s$ conjugate to $r$ along $\gamma$, which is a contradiction.
\end{proof}

\begin{defn}
    Let $\Sigma_t$ be a Cauchy surface in a globally hyperbolic spacetime $(\cM,g)$. The trapped region $\tau_t$ of $\Sigma_t$ is the set of points $p\in \Sigma_t$ for which there exists a trapped surface $S$ with $p\in S \subset \Sigma_t$. The apparent horizon $\mathcal{A}_t$ is the boundary of $\tau_t$.
\end{defn}
Note that the apparent horizon depends on the foliation of the spacetime, which depends on the choice of Cauchy surface $\Sigma_t$. Therefore we must be very careful about trusting numerical results on (weak) cosmic censorship, since one can only determine the apparent horizon and not the event horizon itself.