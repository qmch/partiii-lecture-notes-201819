\begin{note}
There will not be official typed course notes, but there will be scanned handwritten notes, available at \url{www.damtp.cam.ac.uk/user/wingate/AQFT} (Raven login required). Previous lecturers' notes are currently online (Skinner, Osborn).
\end{note}

Today we introduce path integrals in a QFT context. There are some benefits to working with path integrals-- some computations are simplified or more straightforward, and Lorentz invariance is manifest (unlike in the canonical formalism).

\subsection*{Path integrals in quantum mechanics} Rather than trying to tackle the full machinery of QFT, we'll start with $0+1$ dimensional non-relativistic quantum mechanics (cf. Osborn $\mathsection$ 1.2. We'll set $\hbar = 1$ for now, though we may restore it later in order to make arguments when $\hbar \ll 1$ in a classical limit. In these units,
\begin{equation*}
    [E][t]=[\hbar]=[p][x]
\end{equation*}
using uncertainty relations.

Let us consider a Hamiltonian in $1$ spatial dimension,
\begin{equation*}
    \hat H=H(\hat x, \hat p)\quad \text{with }[\hat x,\hat p]=i.
\end{equation*}
We'll further assume for simplicity that the Hamiltonian has a kinetic term and a potential based only on position,
\begin{equation*}
    \hat H= \frac{\hat p^2}{2m}+V(\hat x).
\end{equation*}

Now the Schr\"odinger equation takes the form
\begin{equation}
    i\P{}{t}\ket{\psi(t)}=\hat H \ket{\psi(t)}
\end{equation}
which has formal solution
\begin{equation}\label{schrodingerformal}
    \ket{\psi(t)}=e^{-i\hat H t}\ket{\psi(0)}.
\end{equation}
Let us consider some position eigenstates $\ket{x,t}$ such that
\begin{equation*}
    \hat x(t)\ket{x,t}=x \ket{x,t},\quad x\in \RR,
\end{equation*}
where these states obey some normalization
\begin{equation*}
    \braket{x',t}{x,t}=\delta(x'-x).
\end{equation*}
In the Schr\"odinger picture, states depend on time, while operators are constant. In terms of fixed (time-independent) eigenstates $\set{\ket{x}}$ of the position operator $\hat x$, we may write the wavefunction as
\begin{equation}
    \psi(x,t)=\braket{x}{\psi(t)},
\end{equation}
so that applying the Hamiltonian to the wavefunction $\psi(x,t)$ yields
\begin{equation}
    \hat H \psi(x,t)=(-\frac{1}{2m}\frac{\p^2}{\p x^2} + V(x))\psi(x,t).
\end{equation}

This is the traditional presentation of quantum mechanics and the wavefunction. In the path integral formalism, we'll consider a more particle-like treatment, where we express time evolution as a sum over all trajectories (meeting some boundary conditions) appropriately weighted (by an action).

Recall that our formal solution \ref{schrodingerformal} tells us what $\ket{\psi(t)}$ is-- we can therefore rewrite the wavefunction as
\begin{equation}
    \psi(x,t)\bra{x}e^{-i\hat Ht} \ket{\psi(0)}.
\end{equation}
By inserting a complete set of (position eigen)states, $1=\int dx_0 \ket{x_0}\bra{x_0}$, we get
\begin{align*}
    \psi(x,t) &=\int dx_0 \bra{x} e^{-i\hat Ht}\ket{x_0} \braket{x_0}{\psi(0)}\\
        &= \int dx_0\, K(x,x_0; t) \psi(x_0,0),
\end{align*}
where we have defined 
\begin{equation}
    K(x,x_0;t) \equiv \bra{x} e^{-i\hat Ht}\ket{x_0}.
\end{equation}
$K$ is precisely the amplitude for a particle to propagate from $x_0$ to $x$ in time $t$, and it is this amplitude we will rewrite as a \term{path integral}.

Let us further consider time evolution in discrete steps, with $0\equiv t_0 < t_1 < t_2 < \ldots < t_n < t_{n+1} \equiv T$ so that
\begin{equation*}
    e^{-i\hat H T}= e^{-i\hat H(t_{n+1}-t_n)} \ldots e^{-i\hat H (t_1-t_0)}.
\end{equation*}
As before, we insert complete sets of states, finding that our generic time evolution from any $x_0$ to an $x$ of our choosing takes the following form:
\begin{equation}
    K(x,x_0;T) = \int \left[ \prod_{r=1}^n dx_r \bra{x_{r+1}} e^{-i\hat H(t_{r+1}-t_r)}\ket{x_r}\right] \bra{x_1}e^{-i\hat H t_1}\ket{x_0}.
\end{equation}
That is, we integrate over all intermediate positions $x_r$ for each $t_r$. Naturally, $x_{n+1}$ must be $x$.

Let's look at the free theory first to understand what we've done, setting $V(x)=0$. Now this $K_0$ object we've defined takes the form
\begin{equation}
    K_0(x,x'; t)=\bra{x}e^{-i \frac{\hat p^2}{2m} t}\ket{x'}.
\end{equation}
To make sense of this, we now insert a complete set of momentum eigenstates $\ket{p}$ with the normalization
\begin{equation*}
    \int \frac{dp}{2\pi}\ket{p}\bra{p}=1,
\end{equation*}
recalling that $\braket{x}{p}=e^{ipx}$ are simply plane waves. Then
\begin{equation*}
    K_0 (x,x'; t)=\int \frac{dp}{2\pi} e^{-i p^2 t/2m} e^{ip (x-x')}.
\end{equation*}
We can compute this explicitly. Completing the square with a change of variables to $p'=p- \frac{m(x-x')}{t}$, $K_0$ becomes a Gaussian integral,
\begin{align*}
    K_0(x,x';t) &= e^{im(x-x')^2/2t} \int_{-\infty}^\infty \exp \left[-\frac{i(p')^2 t}{2m}\right]\\
    &= e^{im(x-x')^2/2t} \sqrt{\frac{m}{2\pi i t}}.
\end{align*}
Note that as $t\to 0$,%
    \footnote{This was more obvious from the original expression for $K_0$ where $K_0(x,x'; t=0)= \int \frac{dp}{2\pi} e^{ip(x-x')}.$}
\begin{equation*}
    \lim_{t\to 0} K_0(x,x'; t)=\delta(x-x'),
\end{equation*}
which agrees with the fact that $\braket{x'}{x}=\delta(x-x')$.

For $V(\hat x)\neq 0$, we still need small time steps but since operators generically do not compute, exponentials don't add in the usual way:
\begin{equation*}
    e^{\hat A} e^{\hat B} = \exp(\hat A + \hat B + \frac{1}{2}[\hat A, \hat B] + \ldots) \neq e^{\hat A + \hat B}\quad \text{when }[\hat A, \hat B] \neq 0.
\end{equation*}
This is the Baker-Campbell-Hausdorff (BCH) formula. However, for small $\epsilon$ we can write
\begin{equation*}
    e^{\epsilon \hat A} e^{\epsilon \hat B} = \exp(\epsilon \hat A + \epsilon \hat B +O(\epsilon^2)),
\end{equation*}
or equivalently
\begin{equation*}
    e^{\epsilon(\hat A + \hat B)} = e^{\epsilon \hat A} e^{\epsilon \hat B}(1+O(\epsilon^2)),
\end{equation*}
so we conclude that
\begin{equation*}
    e^{\hat A +\hat B}=\lim_{n\to \infty} \left(e^{\hat A/n}e^{\hat B/n}\right)^n.
\end{equation*}

Suppose now that we divide the overall time $T$ into $n$ equal time steps so that $t_{r-1}-t_r=\delta t$, with $T=n\delta t$. Then one of the intermediate time evolution steps looks like
\begin{align*}
    \bra{x_{r+1}} e^{-i\hat H \delta t}\ket{x_r}
        &= e^{-iV(x_r)\delta t} \bra{x_{r+1}} e^{-i\hat p^2 \delta t/2m} \ket{x_r}\\
        &= \sqrt{\frac{m}{2\pi i \delta t}} \exp\left[\frac{i}{2} m\left(\frac{x_{r+1}-x_r}{\delta t}\right)^2 \delta t - i V(x_r)\delta t\right].
\end{align*}
Substituting this back into the definition of $K(x,x_0;T)$, we find that
\begin{equation}
    K(x,x_0; T) =\int \left( \prod^n_{r=1} dx_r\right) \left(\frac{m}{2\pi i \delta t}\right)^{\frac{n+1}{2}} \exp \left( i\sum_{r=0}^n \left[\frac{m}{2}\left(\frac{x_{r+1}-x_r}{\delta t}\right)^2 -V(x_r)\right]\delta t\right).
\end{equation}
Now we take the limit as $n\to \infty, \delta t\to 0$ with $T$ fixed. Then the argument of the exponential becomes
\begin{equation}
    \int_0^T \bkt{\frac{m}{2} \dot x^2 - V(x)} dt = \int_0^T L dt,
\end{equation}
where $L(x,\dot x)$ is the classical Lagrangian and this integral is nothing more than the action. We conclude that
\begin{equation}
    K(x,x_0;T) = \bra{x} e^{-i\hat H T}\ket{x_0} = \int \mathcal{D}x\, e^{iS[x]},
\end{equation}
where $S[x]=\int_0^T L(x, \dot x) dt$ is the classical action and the $\mathcal{D}$ conceals all our sins (the continuum limit) in a cute integration measure. Note that the action has units of energy $\times$ time, so if we restore $\hbar$, we see that this integral becomes
\begin{equation}
    K(x,x_0 ; T) = \int \mathcal{D}x\, e^{iS/\hbar},
\end{equation}
and in the $\hbar \to 0$ limit (the classical limit), the integral is dominated by paths $x$ which minimize the classical action, which we recognize as Hamilton's principle from classical mechanics.