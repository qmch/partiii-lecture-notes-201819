Last time, we wrote down an action for the electromagnetic field,
\begin{equation*}
    S_g[A]=\frac{1}{4}\int d^4x F_{\mu\nu}F^{\mu\nu} = \frac{1}{2}\int\frac{d^4k}{(2\pi)^4} \tilde A_\mu(-k)(k^2\delta^{\mu\nu}-k^\mu k^\nu)\tilde A_\nu(k),
\end{equation*}
We introduced the Faddeev-Popov method for fixing the gauge. We write the identity in terms of a delta function and an (as yet unspecified) functional $G[A]$,
\begin{equation*}
    1= \int \cD \alpha(x) \delta (G[A]) \det \paren{\frac{\delta G[A]}{\delta \alpha}}.
\end{equation*}

For the $G[A]$ we will use, the determinant (Jacobian factor) will be independent of $A$, though this will not be true more generally in non-Abelian gauge theories. We can denote the gauge-transformed $A$ as
\begin{equation}
    A_\mu^\alpha(x)=A_\mu (x) +\frac{1}{e} \p_\mu \alpha(x).
\end{equation}
If we choose to work in Lorenz gauge, for example, then
\begin{equation}
    G[A]=\p_\mu A^\mu \implies G[A^\alpha]=\p_\mu A^\mu +\frac{1}{e} \p^2 \alpha,
\end{equation}
so
\begin{equation}
    \det \paren{\frac{\delta G[A]}{\delta \alpha}}=\det \paren{\frac{\p^2}{e}}.
\end{equation}
%what does this mean, notationally?
Thus we rewrite the path integral with our delta function as
\begin{align}
    \int \cD A e^{-S_g[A]} &= \det \paren{\frac{\delta G[A^\alpha]}{\delta \alpha}} \int \cD \alpha \int \cD A e^{-S_g[A]} \delta(G[A])\\
    &=\det \paren{\frac{\delta G[A^\alpha]}{\delta \alpha}} \paren{\int \cD \alpha} \int \cD A e^{-S_g[A]} \delta(G[A]),
\end{align}
where we have changed variables $A\mapsto A^\alpha$, and $S_g$ is invariant since this is a gauge transformation.%
    \footnote{n.b. this path integral over gauge transformations $\int \cD \alpha$ is infinite. It is the same infinity that cropped up when we tried to naively compute the full path integral before gauge fixing, but here we have isolated it with our gauge fixing procedure so it can be readily discarded.}

To fix the gauge, let's choose the functional
\begin{equation}
    G[A]=\p_\mu A^\mu(x) -\omega(x) \implies G[A^\alpha]=\p_\mu A^\mu - \omega +\frac{1}{e} \p^2 \alpha.
\end{equation}
We now integrate over $\omega$ with a Gaussian weight factor of mean $0$ and variance $\xi$. Thus
\begin{equation}
    \int \cD \omega \exp \bkt{-\int d^4x \frac{\omega^2}{2\xi}} \det \paren{\frac{\p^2}{e}} \int \cD \alpha \cD A e^{-S[A]}\delta(\p_\mu A^\mu -\omega),
\end{equation}
which becomes (similar to before)
\begin{equation}
    \det \paren{\frac{\p^2}{e}} \paren{\int \cD \alpha} \int \cD A e^{-S_g[A]}\exp \bigg(-\underbrace{\int d^4x \frac{1}{2\xi}(\p_\mu A^\mu)^2}_{S_{gf}[A]}\bigg),
\end{equation}
where $S_{gf}$ indicates a gauge-fixing action. Thus
\begin{equation}
    S_g[A]+S_{gf}[A]=\frac{1}{2} \int \frac{d^4k}{(2\pi)^k} \tilde A_\mu(-k) \bkt{k^2 \delta^{\mu\nu}-(1-\frac{1}{\xi})k^\mu k^\nu} \tilde A_\nu(k),
\end{equation}
where we've just taken a Fourier transform as usual. The propagator solves
\begin{equation}
    (k^2 \delta^{\mu\nu}-(1-\frac{1}{\xi})k^\mu k^\nu)\tilde D_{\nu\rho}(k)=\delta^\mu{}_\rho,
\end{equation}
so the photon propagator takes the form
\begin{equation}
    \tilde D^{\mu\nu}(k)=\frac{1}{k^2}\paren{ \delta^{\mu\nu}-(1-\xi) \frac{k^\nu k^\nu}{k^2}}.
\end{equation}
Here, $\xi=0$ is known as Lorenz or Landau gauge depending on when the $\xi$ condition is imposed, while $\xi=1$ is known as Feynman gauge.

\subsection*{Free fermions (electrons)}
Let us consider an action in terms of fermions,
\begin{equation}
    S[\psi,\bar \psi]=\int d^4x (-\bar \psi(\slashed{\p}+m)\psi),
\end{equation}
where we work in Euclidean signature, $\slashed{\p}=\gamma_\mu \p^\mu$, and the anticommutation relations hold,
\begin{equation}
    \set{\gamma_\mu,\gamma_\nu}=2\delta_{\mu\nu}.
\end{equation}
Our gamma matrices are hermitian, $\gamma_\mu^\dagger =\gamma_\mu$, and our $\gamma_5$ is taken to be $\gamma_5=\gamma_1 \gamma_2 \gamma_3 \gamma_4$. For example,
\begin{equation}
    \gamma_j=\begin{pmatrix}
        0& -i\sigma_j\\
        i\sigma_j & 0
    \end{pmatrix},
    \quad
    \gamma_4 = \begin{pmatrix}1 & 0 \\ 0 &-1\end{pmatrix}
    \text{ or } \begin{pmatrix}0 & 1 \\ 1 & 0\end{pmatrix}.
\end{equation}

We take the Fourier transform of our fermionic fields using
\begin{gather}
    \psi(x) =\int_p e^{ip\cdot x} \tilde \psi(p),\\
    \bar \psi(x) = \int_p e^{-ip\cdot x} \tilde{\bar \psi}(p)
\end{gather}
where $\int_p = \int \frac{d^4p}{(2\pi)^4}$. Thus in Fourier space our action takes the form
\begin{equation}
    S[\tilde \psi, \tilde{\bar \psi}]= \int_p \tilde{\bar \psi}(i\slashed{p}+m)\tilde \psi.
\end{equation}
Adding sources $\tilde \eta, \tilde{\bar \eta}$, the generating functional is then
\begin{align}
    Z[\tilde \eta,\tilde{\bar \eta}]&=\int \cD \tilde \psi \cD \tilde{\bar \psi} \exp \paren{-\int_p \bkt{ \tilde{\bar \psi}(i\slashed{p}+m)\tilde \psi - \tilde{\bar \eta} \tilde \psi +\tilde {\bar \psi}\tilde \eta}}\\
    &= Z[0,0] \exp\paren{-\int_p \tilde{\bar \eta}(i\slashed{p}+m)^{-1} \tilde \eta},
\end{align}
where we have completed the square as usual.

\subsection*{Feynman rules} In addition to the propagators and vertices, fermion loops pick up minus signs. For instance, the position space propagator takes the form
\begin{equation}
    S_F^{\alpha\beta}(x-y)=\avg{\psi^\alpha(x) \bar \psi^\beta(y)}=\int_p e^{ip \cdot(x-y)} \paren{\frac{1}{i \slashed{p}+m}}^{\alpha\beta},
\end{equation}
where $\alpha,\beta$ are spin indices $1,\ldots,4$. If we expand the action $e^{-S_{QED}}$ to second order in the electron-photon coupling, we find terms
\begin{align*}
    (-ie)^2 \avg{\paren{\int d^4x \bar \psi \slashed{A} \psi}\paren{\int d^4y \bar \psi \slashed{A} \psi}} &= (-ie)^2 \int d^4x d^4 y \,
    \avg{\slashed{A}^{\alpha\beta}(x) \slashed{A}^{\gamma\delta}(y) \bar \psi^\alpha(x) \psi^\beta(x) \bar \psi^\gamma(y) \psi^\delta(y)}.
\end{align*}
In general, we need to anticommute the $\psi$s and $\bar \psi$s to form propagators. One term is
\begin{equation}
    -(-ie)^2\int d^4x d^4y \,
    \paren{
        \slashed{A}^{\alpha\beta}(x) \slashed{A}^{\gamma\delta}(y) \underbrace{\psi^\beta(x) \bar \psi^\gamma(y)}_{S_F^{\beta\gamma}(x-y)} \underbrace{\psi^\delta(y) \bar \psi^\alpha(x)}_{S_F^{\delta \alpha}(y-x)}
    }
\end{equation}
where the overall minus sign comes from anticommuting and we've recognized the pairs $\psi \bar \psi$ as propagators.

We get some Feynman rules for this theory:
%see diagram
\begin{enumerate}
    \item The fermion propagator is an oriented line, with $\tilde S_F(p)=\frac{1}{i\slashed{p}+m}$
    \item The photon propagator is a squiggly line, $\tilde D^{\mu\nu}(k)=\frac{1}{k^2}(\delta^{\mu\nu}-(1-\xi)\frac{k^\mu k^\nu}{k^2})$
    \item The vertex gets a $-ie\gamma^\mu$.
    \item We pick up an overall factor of $(-1)^{l_F},$ where $l_F$ is the number of fermion loops.
\end{enumerate}