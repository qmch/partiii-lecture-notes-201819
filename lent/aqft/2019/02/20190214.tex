\subsection*{Dimensional regularization (``dim reg'')}
We would like to understand the behavior of our QFT in different dimensions. Though it may seem like a strange construction, let us work in $d$ dimensions where
\begin{equation}
    d=4-\epsilon\text{, with }0\leq \epsilon < 1.%
        \footnote{Compare the epsilon expansion from David Tong's \emph{Statistical Field Theory} notes.}
\end{equation} 
We take the action of $\phi^4$ theory,
\begin{equation}
    S=\int d^dx \bkt{ \frac{1}{2}(\p \phi)^2 + \frac{1}{2} m^2 \phi^2 + \frac{\lambda}{4!} \phi^4}.
\end{equation}
Working in $\hbar=c=1$ units, let's analyze the mass dimension of the different quantities and couplings in this action. Given that the action must be dimensionless, $[S]=0$, we get $[\p]=[m]=[x^{-1}]=1$. Recall that in 4 dimensions, $\lambda$ was a dimensionless coupling and therefore ``marginal.'' From the mass term we have 
\begin{equation}
    [m^2\phi^2]=2[m]+2[\phi]=d \implies [\phi]\frac{d}{2}-1,
\end{equation} 
which means our $\lambda$ term now has dimensions given by
\begin{equation}
    [\lambda \phi^4]=d \implies [\lambda] =4-d=\epsilon.
\end{equation}
Thus $\lambda$ is no longer dimensionless. Let us now introduce an arbitrary mass scale $\mu$ and a new dimensionless coupling $g$ such that 
\begin{equation}
    \lambda = \mu^\epsilon g(\mu),
\end{equation}
If we return to our one-loop graph, we see that the self-energy $\Pi^{\text{1-loop}}$ is
\begin{equation}\label{epsilonexpansionintegral}
    \Pi^{\text{1-loop}} =-\frac{1}{2} g(\mu) \mu^\epsilon \int \frac{d^dk}{(2\pi)^d} \frac{1}{k^2+m^2} =-\frac{1}{2} g(\mu) \mu^\epsilon \frac{S_d}{2(2\pi)^d} \int_0^\infty \frac{k^{d-1}dk}{k^2+m^2},
\end{equation}
with $S_d$ the surface area of the $d$-sphere. Note that we can actually define the surface area of the unit sphere in $d$ dimensions even when $d$ is not an integer:
\begin{equation}
    (\sqrt{\pi})^d = \int \prod_{i=1}^d e^{-x_i^2} dx_i = S_d \int_0^\infty e^{-r^2} r^{d-2}dr =\frac{S_d}{2} \Gamma(d/2).
\end{equation}
Thus
\begin{equation}
    S_d = \frac{2\pi^{d/2}}{\Gamma(d/2)},
\end{equation}
which we take to be the definition of $S_d$ in $d\in \CC$ dimensions. We can also analytically continue the $\Gamma$ function by establishing a recursion relation:
\begin{align*}
    \Gamma(\alpha)&=\int_0^\infty dx\, x^{\alpha-1} e^{-x}\\
    &=\frac{1}{\alpha}\bkt{x^{-\alpha}e^{-x}}_0^\infty +\frac{1}{\alpha} \int_0^\infty dx\, x^\alpha e^{-x}\\
    &= 0 +\frac{1}{\alpha}\Gamma(\alpha+1)
\end{align*}
where the first term is zero if $\text{Re}(\alpha)>0$. We analytically continue and define $\Gamma(\alpha)$ for $\text{Re}(\alpha)>-1$ through 
\begin{equation}
    \Gamma(\alpha)=\frac{1}{\alpha}\Gamma(\alpha+1).
\end{equation}
Note that there are poles when $\text{Re}(\alpha)\in \ZZ^- \cup \set{0}.$

Finally, note that there is an expansion of $\Gamma$ for small $\alpha$: it is
\begin{equation}
    \log \Gamma(\alpha+1)=
    -\gamma \alpha -\sum_{k=2}^\infty (-1)^k \frac{1}{k} \zeta(k) \alpha^k
\end{equation}
where $\gamma\approx 0.577216\ldots$ is the Euler-Mascheroni constant and $\zeta$ is the Riemann zeta function,
\begin{equation*}
    \zeta(k)=\sum_{n=1}^\infty\frac{1}{n^k}.
\end{equation*}
We will usually use this to write
\begin{equation}
    \Gamma(\epsilon)=\frac{1}{\epsilon}-\gamma+O(\epsilon)\text{ for small }\epsilon.
\end{equation}

It is also useful to note that there exists an Euler ``beta function'' given by
\begin{equation}
    B(s,t) =\int_0^a du\, u^{s-1}(1-u)^{t-1} = \frac{\Gamma(s) \Gamma(t)}{\Gamma(s+t)}.
\end{equation}

Now let us return to our integral \ref{epsilonexpansionintegral}. Using these facts, it becomes
\begin{align*}
    \mu^\epsilon \int_0^\infty \frac{k^{d-1}dk}{k^2+m^2} &= \frac{\mu^\epsilon}{2} \int_0^\infty \frac{(k^2)^{d/2-1}dk^2}{k^2+m^2}\\
    &= \frac{m^2}{2} \paren{\frac{\mu}{m}}^{\epsilon} \int_0^1 du(1-u)^{d/2-1} u^{-d/2}\\
    &= \frac{m^2}{2} \paren{\frac{\mu}{m}}^\epsilon \frac{\Gamma(d/2) \Gamma(1-d/2)}{\Gamma(1)},
\end{align*}
where we have used the substitution $u=\frac{m^2}{k^2+m^2}$. Therefore
\begin{equation}
    \Pi^{\text{1-loop}}=-\frac{g(\mu)m^2}{2(4\pi)^{d/2}} \paren{\frac{\mu}{m}}^\epsilon \Gamma(1-d/2).
\end{equation}
Using the recursion relation, we have
\begin{equation}
    \Gamma(1-d/2)=\Gamma(\epsilon/2-1)=-\frac{1}{(1-\epsilon/2)}\Gamma(\epsilon/2).
\end{equation}
We can now unpack some of the factors as
\begin{equation}
    \paren{\frac{4\pi\mu^2}{m^2}}^{\epsilon/2} = 1+\frac{\epsilon}{2} \log\paren{\frac{4\pi \mu^2}{m^2}}+O(\epsilon^2).
\end{equation}
Using these last two expressions, we have
\begin{equation}
    \Pi^{\text{1-loop}}=-\frac{g(\mu)m^2}{32\pi^2} \bkt{\frac{2}{\epsilon} -\gamma+1 +\log \paren{\frac{4\pi\mu^2}{m^2}}}+O(\epsilon).
\end{equation}

We see that the old $\Lambda\to\infty$ divergence appears here as a $1/\epsilon$ pole. In order to make this contribution converge, we must add a counterterm $\frac{\delta m^2}{2} \phi^2$. The first thing we might think to do is the \term{minimal subtraction} (MS) scheme,
\begin{equation}
    \delta m^2 =-\frac{g(\mu)m^2}{16\pi^2 \epsilon},
\end{equation}
by which we just get rid of the epsilon divergence directly.
There's also the \term{modified minimal subtraction scheme} ($\overline{\text{MS}}$), where we also get rid of the extra constants hanging around,
\begin{equation}
    \delta m^2 = -\frac{g(\mu)m^2}{32\pi^2} \paren{\frac{2}{\epsilon}-\gamma+\log 4\pi}.
\end{equation}
%which is used widely in practice. 
%We might also consider an on-shell scheme, with $m^2+\delta m^2=m_{\text{phys}}^2$. 
After this regularization, we have
\begin{equation}
    \Pi^{\text{1-loop},\overline{MS}}=\frac{g(\mu)m^2}{32\pi^2} \paren{\log \frac{\mu^2}{m^2}-1}.
\end{equation}
We can also renormalize the quartic term $\frac{\lambda}{4!}\phi^4$ in an equivalent way by writing out the Feynman loop diagrams and calculating the self-energy from these guys. We'll get amplitudes like
\begin{equation}
    \frac{\lambda^2}{2}\int \frac{d^4k}{(2\pi)^4} \frac{1}{k^2+m^2} \frac{1}{(p_1+p_2+k)^2 +m^2}+\text{ two similar terms}.
\end{equation}
The one-loop contribution to $g(\mu)=\lambda \mu^{-\epsilon}$ is
\begin{equation}
    \frac{g^2 \mu^\epsilon}{2}\int \frac{d^dk}{(2\pi)^d} \frac{1}{k^2+m^2} \frac{1}{(p_1+p_2+k)^2+m^2}+t,u\text{-channel diagrams}.
\end{equation}
If we set $p_i=0$ to find the pure $\phi^4$ coupling in the effective action $\Gamma[\phi]$, we get an integral that looks like
\begin{equation}
    \mu^\epsilon \int_0^\infty \frac{k^{d-1} dk}{(k^2+m^2)^2} = \frac{1}{2} \paren{\frac{\mu}{m}}^\epsilon \frac{\Gamma(2-d/2)\Gamma(d/2)}{\Gamma(2)}.
\end{equation}
and we get the result
\begin{equation}
    \frac{3g^2 S_4}{2(2\pi)^d} \frac{1}{2}\paren{\frac{\mu}{m}}^\epsilon \frac{\Gamma(2-d/2)\Gamma(d/2)}{\Gamma(2)}=\frac{3g^2}{32\pi^2} \paren{\frac{2}{\epsilon}-\gamma +\log \frac{4\pi \mu^2}{m^2}} +O(\epsilon).
\end{equation}
Introducing a counterterm $\delta g=\frac{3g^2}{32\pi^2} \paren{\frac{2}{\epsilon}-\gamma+\log 4\pi}$ (the $\overline{\text{MS}}$ scheme), we get an effective coupling
\begin{equation}
    g_{\text{eff}}(\mu)=g(\mu)-\frac{3g^2(\mu)}{32\pi^2} \log \frac{\mu^2}{m^2}+\ldots
\end{equation}